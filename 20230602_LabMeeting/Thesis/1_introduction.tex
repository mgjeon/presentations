% !TEX root = main.tex

\section{Introduction}
The solar activities, such as flares and coronal mass ejections, can influence the Earth. Understanding these solar phenomena requires accurate knowledge of the coronal magnetic fields, as they play a crucial role in governing such activities. Stokes polarimetry, based on the Zeeman effect, has proven to be an effective method for observing solar magnetic fields \parencite{stenflo2013solar}. The Helioseismic and Magnetic Imager (HMI) aboard the Solar Dynamics Observatory (SDO) routinely provides high-resolution photospheric vector magnetograms using polarimetry techniques \parencite{schou2012polarization}. However, the measurement of coronal magnetic fields remains challenging due to the difficulty in detecting the Zeeman effect at the extreme temperatures of the solar corona, approximately $10^6$ K \parencite{cargill2009coronal}. Consequently, in most cases, only photospheric vector magnetic fields are available, while the estimation of coronal magnetic fields requires the solution of partial differential equations (PDEs) that describe the physical conditions in the solar corona.

One widely adopted assumption for describing the coronal magnetic field is the force-free assumption, which assumes that the Lorentz force dominates over other forces in the solar corona. This assumption is based on the low plasma beta ($\beta$) in the coronal plasma, which compares the plasma pressure gradient force to the magnetic pressure gradient force \parencite{gary2001plasma}. The force-free assumption leads to a fundamental equation that the static coronal magnetic field must satisfy, known as the force-free equation: $(\nabla \times \mathbf{B})\times\mathbf{B} = 0$. Furthermore, since it is a magnetic field, an additional equation, known as Gauss's law for the magnetic field or the solenoidal condition ($\nabla \cdot \mathbf{B} = 0$), must also be satisfied. Magnetic fields that fulfill the force-free equation are referred to as force-free fields. The force-free equation can also be expressed as $\nabla \times \mathbf{B} = \alpha \mathbf{B}$, where $\alpha$ is called the force-free parameter or force-free function. A constant $\alpha$ throughout the computational domain corresponds to linear force-free fields (LFFFs), while a varying $\alpha$ gives rise to nonlinear force-free fields (NLFFFs). Linear force-free fields with $\alpha = 0$ correspond to potential fields. Due to the spatial variation of the force-free parameter in the solar corona, linear force-free fields are considered incomplete representations of the coronal magnetic field.

The optimization method, one of the traditional numerical methods, has been widely used to reconstruct coronal magnetic fields by solving the force-free equation. This method, originally introduced by \parencite{wheatland2000optimization}, involves minimizing a functional $L$ in \cref{eq:optL}. This functional combines the force-free equation and the solenoidal condition. Recently, there has been a growing interest in an emerging technique called physics-informed neural networks (PINNs) for solving partial differential equations \parencite{karniadakis2021physics}. Physics-informed neural networks use the loss function derived from the partial differential equations and aim to minimize it to obtain the numerical solution. Physics-informed neural networks offer several advantages, such as their capability to handle complex functions, flexibility, and the ability to compute derivatives using automatic differentiation.

The objective of this study is to investigate the performance and capabilities of the physics-informed neural network method in calculating coronal magnetic fields. I utilize the same loss function $L$ employed in the traditional optimization method, with a specific focus on its application to the Low-Lou model. Through this investigation, I seek to evaluate the potential of the physics-informed neural network method as a novel computational tool for studying solar magnetic fields.