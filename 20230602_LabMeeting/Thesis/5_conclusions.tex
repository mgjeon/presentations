% !TEX root = main.tex

\section{Conclusions}


The findings of this study yield valuable insights into the performance and capabilities of the physics-informed neural network method in calculating nonlinear force-free fields. The evaluation of various metrics demonstrates the PINN approach's ability to accurately reconstruct the Low-Lou model. The majority of these metrics demonstrate values that are quite similar with the analytical solutions, highlighting a substantial level of agreement between the PINN-reconstructed fields and the analytical solutions (\Cref{tab:metric} and \Cref{fig:metric}). Furthermore, this agreement is qualitatively confirmed by observing the graphical representation of the magnetic field (\Cref{fig:xy,fig:yz,fig:xz,fig:xz_tilted}). Additionally, the PINN approach successfully surpasses the potential field in terms of proximity to the Low-Lou solution, highlighting its effectiveness in capturing the complex magnetic field structures associated with solar activities.

While the PINN approach proves to be a promising method for calculating nonlinear force-free fields, this study also sheds light on its limitations. The figures of merit for the PINN(50000) model are inferior to those of the PINN(25000) model, despite the former having a lower loss value (\Cref{tab:metric}). This indicates the need for further exploration and refinement of loss functions. Additionally, the magnetic energy of the PINN(50000) model is lower than that of the analytical solution, suggesting that the PINN exhibits a spectral bias and struggles to learn high-frequency features. Enhancing the loss function and neural network architecture could improve the PINN's ability to capture high-frequency features and achieve even closer agreement with the analytical solutions. This research underscores the importance of ongoing efforts to enhance the performance and accuracy of the PINN approach, allowing for its wider application in solar physics and computational magnetohydrodynamics.

This thesis has presented a novel approach to calculating nonlinear force-free fields by employing physics-informed neural networks. This research specifically focuses on the application of physics-informed neural networks to the Low-Lou model, which provides semi-analytical solutions for nonlinear force-free fields. The results demonstrate the effectiveness of the proposed approach in accurately computing these fields. The successful implementation of physics-informed neural networks in this context highlights their versatility and potential for further advancements in the study of coronal magnetic fields and their relationship to solar phenomena. By recognizing the significance of coronal magnetic fields, this study demonstrates the potential of physics-informed neural networks in solar physics research.